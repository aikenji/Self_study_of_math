\section{Language of set theory}

We construct a formal language suitable for describing sets. The language consists of some mathematical symbols as well as purely logical symbols.

The complete list of symbols of language is as below:
\begin{definition}{Symbols in LOST(Language of Set Theory)}{}
    \begin{enumerate}
        \item variable: $v_0, v_1, v_2, \ldots$
        \item equality: $=$
        \item membership: $\in$
        \item connectives: $\neg, \land, \lor, \implies, \iff$
        \item quantifiers: $\forall, \exists$
        \item parentheses: $\left(,\ \right)$
    \end{enumerate}
\end{definition}

\begin{remark}
    Bounded set quantifiers shall be used. Thus, we can abbreviate the formula\\
    $\forall x (x \in y \implies x \notin a)$ by $(\forall x \in y) (x \notin a).$\\
    $\exists x (x \in y \land x \notin a)$ by $(\exists x \in y) (x \notin a).$\\
\end{remark}

\begin{definition}{Zermelo-Fraenkel Axioms}{}
   \begin{enumerate}
        \item \textbf{Extensionality Axiom.}\\
        Two sets are equal iff they have the same elements.\\
        \\
        $\forall A \forall B \left(A=B \iff \forall x \left(x \in A \iff x \in B\right)\right)$ \\
        \item \textbf{Empty Set Axiom.} \\
        There is a set with no elements.\\
        \\
        $\exists A \forall x (x \notin A)$\\
        \item \textbf{Subset Axiom.} \\
        Let $\varphi(x)$ be a formula. For every set A there exists a set S that consists of all $x \in A$ with $\varphi(x)$ holds.\\
        \\
        $\forall A \exists S \forall x (x \in S \iff (x \in A \wedge \varphi(x))) $\\
        \item \textbf{Pairing Axiom.}\\
        For every u and v there is a set that consists of just u and v.\\
        \\
        $\forall u \forall v \exists S \forall x (x \in S \iff (x = u \vee x = v)) $\\
        
        \item \textbf{Union Axiom.}\\
        For every set $\mathcal{F}$ there exists a set U that consists of all elements that belong to at least one set in $\mathcal{F}$.\\
        \\
        $\forall \mathcal{F} \exists U \forall x (x \in U \iff \exists F (x \in F \wedge F \in \mathcal{F}))$\\
        \item \textbf{Power Set Axiom.}\\
        For every set A there is a set P that consists of all subsets of A.\\
        \\
        $\forall A \exists \mathcal{P} \forall P (P \in \mathcal{P} \iff P \subseteq A)$
   \end{enumerate} 
\end{definition}

\begin{remark}
    All sets guaranteed by axioms 2-6 are unique. \\
    \begin{enumerate}
        \item The empty set axiom defines the empty set denoted by $\varnothing$.\\
        \item The subset axiom defines the set denoted by $\{x \in A:\varphi(x)\}$.\\
        \item The pairing axiom defines the unordered pair set denoted by $\{u, v\}$.\\
        \item The union axiom defines the union of $\mathcal{F}$ denoted by $\cup \mathcal{F}$.\\
        \item The power set axiom defines the power set denoted by $\mathcal{P}(A) = \{X:X \subseteq A\}.$
    \end{enumerate}
\end{remark}

\begin{definition}{Class}{}
    We shall refer to any collection of the form $\{x:\varphi(x)\}$ as a \textbf{class}. When the class is not a set, then we call it \textbf{proper class}, such as $\{x:x=x\}$, sometimes we also call it unbounded collection.
\end{definition}

\begin{theorem}{}{}
    Let $\varphi(x)$ be a formula. Suppose that there is a set A s.t. for all $x$, if $\varphi(x)$, then $x \in A$.\\
    Then there is a unique set S s.t. for all $x$, $x \in S \iff \varphi(x)$. \\
    \\
    In other words, the class $\{x:\varphi(x)\}$ is, in fact, equal to the set S.
\end{theorem}

\begin{corollary}{}{}
    \begin{enumerate}
        \item Intersection: $A \cap B = \{x:x \in A \land x \in B\}$\\
        \item Difference: $A \setminus B = \{x:x \in A \land x \notin B\}$ \\
    \end{enumerate}
    By the previous theorem, these set operations are well defined and create new sets.
\end{corollary}
