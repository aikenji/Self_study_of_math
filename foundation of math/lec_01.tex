\section{Language of set theory}

\subsection{Basic Set-Building Axioms}

We construct a formal language suitable for describing sets. The language consists of some mathematical symbols as well as purely logical symbols.

The complete list of symbols of language is as below:
\begin{definition}{Symbols in LOST(Language of Set Theory)}{}
    \begin{enumerate}
        \item variable: $v_0, v_1, v_2, \ldots$
        \item equality: $=$
        \item membership: $\in$
        \item connectives: $\neg, \land, \lor, \implies, \iff$
        \item quantifiers: $\forall, \exists$
        \item parentheses: $\left(,\ \right)$
    \end{enumerate}
\end{definition}

\begin{remarks}
    Bounded set quantifiers shall be used. You can abbreviate the formula:
    \begin{enumerate}
        \item $\forall x (x \in y \implies x \notin a)$ by $(\forall x \in y) (x \notin a).$
        \item $\exists x (x \in y \land x \notin a)$ by $(\exists x \in y) (x \notin a).$
    \end{enumerate}
\end{remarks}

\begin{definition}{Zermelo-Fraenkel Axioms}{}
   \begin{enumerate}
        \item \textbf{Extensionality Axiom.}\\
        Two sets are equal iff they have the same elements.\\
        \\
        $\forall A \forall B (A=B \iff \forall x (x \in A \iff x \in B))$ \\
        \item \textbf{Empty Set Axiom.} \\
        There is a set with no elements.\\
        \\
        $\exists A \forall x (x \notin A)$\\
        \item \textbf{Subset Axiom.} \\
        Let $\varphi(x)$ be a formula. For every set $A$ there exists a set $S$ that consists of all $x \in A$ with $\varphi(x)$ holds.\\
        \\
        $\forall A \exists S \forall x (x \in S \iff (x \in A \wedge \varphi(x))) $\\
        \item \textbf{Pairing Axiom.}\\
        For every $u$ and $v$ there is a set that consists of just $u$ and $v$.\\
        \\
        $\forall u \forall v \exists S \forall x (x \in S \iff (x = u \vee x = v)) $\\
        
        \item \textbf{Union Axiom.}\\
        For every set $\mathcal{F}$ there exists a set $U$ that consists of all elements that belong to at least one set in $\mathcal{F}$.\\
        \\
        $\forall \mathcal{F} \exists U \forall x (x \in U \iff \exists F \in \mathcal{F} (x \in F))$\\
        \item \textbf{Power Set Axiom.}\\
        For every set $A$ there is a set $\mathcal{P}$ that consists of all subsets of $A$.\\
        \\
        $\forall A \exists \mathcal{P} \forall P (P \in \mathcal{P} \iff P \subseteq A)$
   \end{enumerate} 
\end{definition}

\begin{remarks}
    \begin{enumerate}
        \item The empty set axiom defines the \textbf{empty set} denoted by $\varnothing$.
        \item The subset axiom defines the set denoted by $\{x \in A:\varphi(x)\}$.
        \item The pairing axiom defines the \textbf{unordered pair} denoted by $\{u, v\}$. If $u = v$, then the set $\left\{u\right\}$ is referred to as a \textbf{singleton}.
        \item The union axiom defines the \textbf{union} of $\mathcal{F}$ denoted by $\bigcup \mathcal{F}$.
        \item The power set axiom defines the \textbf{power set} of $A$ denoted by $\mathcal{P}(A) = \{X:X \subseteq A\}.$
    \end{enumerate}
\end{remarks}

\begin{definition}{Class}{}
    We shall refer to any collection of the form $\{x:\varphi(x)\}$ as a \textbf{class}. We call it \textbf{proper class}, 
    when the class is not a set, such as $\{x:x=x\}$. Sometimes we also call it \textbf{unbounded collection.}
\end{definition}

\begin{theorem}{Sufficient condition for class to be a set}{class}
    Let $\varphi(x)$ be a formula. Suppose that there is a set $A$ such that for every $x$, if $\varphi(x)$, then $x \in A$. Then there is a unique set $S$ such that for all $x$, $x \in S \iff \varphi(x)$.\\
    \begin{equation*}
        \exists A \forall x (\varphi(x) \implies x \in A) \implies  \exists ! S \forall x(x \in S \iff \varphi(x))
    \end{equation*}
    \\
    In other words, the class $\{x:\varphi(x)\}$ is equal to the set $S$.
\end{theorem}

\begin{proof}
    Let $S = \left\{x \in A \colon \varphi(x) \right\}$ which is uniquely defined by subset axiom.\\
    ($ \implies $)\\
    $x \in S \implies x \in A \land \varphi(x) \implies \varphi(x).$\\
    ($ \Leftarrow $)\\
    $\varphi(x) \implies x \in A \implies x \in A \land \varphi(x) \implies x \in S.$
\end{proof}
\begin{remarks} 
    Furthermore, the sufficient and necessary condition for class to be a set is as follows:
    \begin{equation*}
        \exists S \forall x(x \in S \iff \varphi(x)) \iff \exists A \forall x(\varphi(x) \implies x \in A)
    \end{equation*}
\end{remarks}

\begin{corollary}{}{}
    Let $\varphi(x)$ be a formula. Then $\{x:\varphi(x)\}$ is a proper class iff for every set $A$ there is a $x$ s.t. $\varphi(x)$ and $x \notin A$.
    
\end{corollary}
By the \Cref{class}, Some sets operations below are well defined and can be used to create new sets.

\begin{corollary}{}{}
    Union, Intersection and Difference of sets can be defined by \Cref{class}. For any set $A$, $B$ and nonempty collection $\mathcal{F}$:\\
    \begin{enumerate}
        \item Union: $A \cup B = \{x: x \in A \lor x \in B\} $ 
        \item Intersection: $A \cap B = \{x: x \in A \land x \in B\}$
        \item Difference: $A \setminus B = \{x: x \in A \land x \notin B\}$
        \item Union of $\mathcal{F}$: $\bigcup \mathcal{F} = \{x: \exists F \in \mathcal{F}(x\in F)\}$
        \item Intersection of $\mathcal{F}$: $\bigcap \mathcal{F} = \{x: \forall F \in \mathcal{F}(x\in F)\}$
    \end{enumerate}
\end{corollary}

\begin{remark}
    If collection $\mathcal{F}$ is empty, then\\
    \begin{enumerate}
        \item $\bigcup \varnothing = \varnothing$ 
        \item $\bigcap \varnothing$ is not well defined as set. Because the statement $x$ belongs to every members of empty set is vacuously true. So all x will belongs to it, which is a contradiction.
    \end{enumerate}
\end{remark}

\begin{theorem}{}{}
    Suppose nonempty sets $\mathcal{F}$, $\mathcal{G}$ and $\mathcal{F} \subseteq \mathcal{G}$. Then $\bigcup \mathcal{F} \subseteq \bigcup \mathcal{G}$ and $\bigcap \mathcal{G} \subseteq \bigcap \mathcal{F}$.
\end{theorem}

\begin{proof}
    To proof the first one, Let $x \in \bigcup \mathcal{F}$.
    \begin{align*}
        &\implies x \in F \ \text{for some}\ F \in \mathcal{F} \land \mathcal{F} \subseteq \mathcal{G} \\
        &\implies x \in F \ \text{for some}\ F \in \mathcal{G}\\
        &\implies x \in \bigcup \mathcal{G}\\
        &\implies \bigcup \mathcal{F} \subseteq \bigcup \mathcal{G}\\
    \end{align*}
\end{proof}

\newpage

\subsection{Operations on Sets}

There are some other important ways to build new sets by axioms in previous section.

\begin{lemma}{}{}
    Let $A$ and $\mathcal{F}$ be sets. Then there is a unique set $\mathcal{C}$ such that for all $X$,
    \begin{equation*}
        X \in \mathcal{C} \iff X = A \setminus F\ \text{for some}\ F \in \mathcal{F}
    \end{equation*}
    or,
    \begin{equation*}
        X \in \mathcal{C} \iff \exists F \in \mathcal{F}(X = A \setminus F) 
    \end{equation*}
    We shall denote $\mathcal{C}$ by $\{A \setminus F: F \in \mathcal{F}\}$.
\end{lemma}

\begin{proof}
    $\exists F \in \mathcal{F}(X = A \setminus F) \implies X \subseteq A \implies X \in \mathcal{P}(A)$\\
    by \Cref{class}, set $\mathcal{C}$ is uniquely constructed.

    
\end{proof}

\begin{theorem}{De Morgan's Laws}{DML}
    If $A$ is a set and $\mathcal{F}$, then
    \begin{eqnarray*}
        A \setminus \bigcup \mathcal{F} = \bigcap \{A \setminus F: F \in \mathcal{F}\}\\
        A \setminus \bigcap \mathcal{F} = \bigcup \{A \setminus F: F \in \mathcal{F}\}
    \end{eqnarray*}
\end{theorem}

\begin{proof}
    Proof the first one.\\
    \begin{align*}
        x \in A \setminus \bigcup \mathcal{F} &\iff x \in A \land x \notin \bigcup \mathcal{F}\\
        &\iff x \in A \land (\forall F \in \mathcal{F}(x \notin F)) \\
        &\iff \forall F \in \mathcal{F}(x \in A \land x \notin F)\\ 
        &\iff \forall F \in \mathcal{F}(x \in A \setminus F)\\ 
        &\iff x \in \bigcap \{A \setminus F : F \in \mathcal{F}\}
    \end{align*}
\end{proof}

\newpage
