\section{Natural Numbers}

\subsection{Peano Postulates}

\begin{definition}{}{}
    Let $S : N \to N$ and let $A \subseteq N$. Then $A$ is said to be closed under $S$ iff
    \begin{equation*}
        S(x) \in A \text{ for all } x \in A
    \end{equation*}
\end{definition}

\begin{definition}{Peano System}{}
    Let $(N,S,e)$ be an ordered triple that consists of a set $N$, a
    function $S : N \to N$, and an element $e \in N$. Then $(N,S,e)$
    is a \textbf{Peano system} if the following conditions hold:
    \begin{enumerate}

        \item $e \notin ranS$
        \item $S$ is injective.
        \item For all $A \subseteq N$
        \begin{equation*}
            e \in A \land A \text{ is closed under } S \implies A = N
        \end{equation*}

    \end{enumerate}
    
\end{definition}

\subsection{Inductive Sets}

In this section, we will construct natural numbers under the 
architecture of set theory.

\begin{definition}{Successor}{}
    For each set $x$, the \textbf{successor} $x^{+}$ is the set defined by 
    \begin{equation*}
        x^{+} = x \cup \{x\}
    \end{equation*}
\end{definition}

\begin{proposition}{}{}
    \begin{enumerate}

        \item $a \in x^{+} \iff a \in x \lor a = x$
        \item $x \in x^{+}$
        \item $x \subseteq x^{+}$

    \end{enumerate}
\end{proposition}

\begin{example}
   The first few natural numbers as follows
   \begin{itemize}

       \item $0 = \varnothing$
       \item $1 = \{0\}$
       \item $2 = \{0,1\}$
       \item $3 = \{0,1,2\}$
       \item $4 = \{0,1,2,3\}$

   \end{itemize} 
\end{example}

\begin{definition}{Inductive}{}
    A set $I$ is said to be \textbf{inductive} iff
    \begin{enumerate}

        \item $\varnothing \in I$
        \item $\forall a \in I (a^{+} \in I)$

    \end{enumerate}
    The second one can be also restated as "The set is closed under successor."
\end{definition}

\begin{lemma}{Infinity Axiom}{}
    There is a inductive set.
    \begin{equation*}
        \exists I (\varnothing \in I \land \forall x \in I(x^{+} \in I))
    \end{equation*}
\end{lemma}

\begin{definition}{Natural Numbers}{}
    A \textbf{natural number} is an element that belongs to every
    inductive sets. In other words, $x$ is a natural number iff
    $x$ in every inductive sets.
\end{definition}

\begin{definition}{Natural Number Set}{}
    There exits a unique set $\omega$ such that for all $x$
    \begin{equation*}
        x \in \omega \iff x \text{ in every inductive set}
    \end{equation*}
    We denote the set as 
    \begin{equation*}
        \omega = \{x : x \text{ in every inductive set}\}
    \end{equation*}
\end{definition}

\begin{proof}
    By infinity axiom, there exits an inductive set $A$. And
    \begin{equation*}
        x \text{ in every inductive set } \implies x \in A.
    \end{equation*}
    By \Cref{class}, there is a unique set $\omega$ s.t.
    \begin{equation*}
        x \in \omega \iff x \text{ in every inductive set}
    \end{equation*}
\end{proof}

\begin{theorem}{}{}
    The set $\omega$ is inductive and is a subset of every inductive set. Hence $\omega$ is the smallest inductive set. 
\end{theorem}

\begin{corollary}{Principle of Mathematical Induction}{label:PMI}
    If $I$ is inductive and $I \subseteq \omega$, then $I = \omega$.
\end{corollary}
    
\begin{remarks}
    Suppose $P(n)$ is some property. To prove by induction that 
    \begin{equation*}
        \forall n \in \omega P(n)
    \end{equation*}
    We just let $I = \{n \in \omega : P(n)\} \subseteq \omega$ 
    If we can prove that 
    \begin{enumerate}

        \item $0 \in I$
        \item $n \in I \implies n^{+} \in I$

    \end{enumerate} 
    Then $I = \omega$, by \Cref{label:PMI}. Therefore $P(n)$ holds
    for all natural numbers.
\end{remarks}

\begin{theorem}{}{}
    For every $n \in \omega$, either $n = 0$, or $n = k^{+}$ for 
    some $k \in \omega$.
\end{theorem}

\begin{proof}
    Let $I = \{n \in \omega: n = 0 \lor \exists k \in \omega(n = k^{+})\}$\\
    \begin{itemize}

        \item Obviously, $0 \in I$.
        \item Let $n \in I$, then $n = 0 \lor n = k^{+} \text{ for some } k \in \omega$. 
        We get $n^{+} = 0^{+} \lor n^{+} = (k^{+})^{+}$
        where $k^{+} \in \omega$. Thus $n^{+} \in I$.

    \end{itemize}
    By \Cref{label:PMI}, $I = \omega$.
\end{proof}

\begin{theorem}{Recursion Theorem}{}
    
\end{theorem}


